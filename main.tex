\documentclass[12pt]{article}
\usepackage{graphicx} % Required for inserting images

% Refs and Bib
\usepackage{cite}               % order multiple entries in \cite{...}
\usepackage{breakurl}           % break too-long urls in refs
\usepackage{url}                % allow \url in bibtex for clickable links
\usepackage{xcolor}             % color definitions, to be use for...
\usepackage[]{hyperref}         % ...clickable refs within pdf...
\hypersetup{                    % ...like so
  colorlinks,
  linkcolor={green!80!black},
  citecolor={red!70!black},
  urlcolor={blue!70!black}
}

% \usepackage[%
%   backend=bibtex      % biber or bibtex
% %,style=authoryear    % Alphabeticalsch
%  ,style=numeric-comp  % numerical-compressed
%  ,sorting=none        % no sorting
%  ,sortcites=true      % some other example options ...
%  ,block=none
%  ,indexing=false
%  ,citereset=none
%  ,isbn=true
%  ,url=true
%  ,doi=true            % prints doi
%  ,natbib=true         % if you need natbib functions
% ]{biblatex}
% \addbibresource{refs.bib}  % better than \bibliography

% Paragraph indentation control
\usepackage{parskip}

% Papar margins
\usepackage[letterpaper,
            margin=1in,
            bottom=1in]{geometry}

% % Listing preferences
% \usepackage{listings}
% \usepackage{color}

% %% Custom colors
% \definecolor{deepblue}{rgb}{0,0,0.5}
% \definecolor{deepred}{rgb}{0.6,0,0}
% \definecolor{deepgreen}{rgb}{0,0.5,0}

% % Python style for highlighting
% \newcommand\pythonstyle{\lstset{
% language=Python,
% basicstyle=\ttm,
% morekeywords={self},              % Add keywords here
% keywordstyle=\ttb\color{deepblue},
% emph={MyClass,__init__},          % Custom highlighting
% emphstyle=\ttb\color{deepred},    % Custom highlighting style
% stringstyle=\color{deepgreen},
% frame=tb,                         % Any extra options here
% showstringspaces=false
% }}
\usepackage{pythonhighlight}


% % Python environment
% \lstnewenvironment{python}[1][]
% {
% \pythonstyle
% \lstset{#1}
% }
% {}

% % Python for external files
% \newcommand\pythonexternal[2][]{{
% \pythonstyle
% \lstinputlisting[#1]{#2}}}

% % Python for inline
% \newcommand\pythoninline[1]{{\pythonstyle\lstinline!#1!}}

% Timeline
\usepackage{tikz}
\usetikzlibrary{fit, calc, decorations.markings}

% TODO: Change title to reflect changes in project.

% Title
% \title{Knowledge Graph Augmented Generative AI for Course Recommendation and Schedule Building}
% 
%% make title bold and 14 pt font (Latex default is non-bold, 16 pt)
\title{\Large \bf 
Knowledge Graph Augmented Generative AI for Course Recommendation and Schedule Building
}

% Author(s) information
% \author{Adebayo Braimah}
\author{
{\rm Adebayo Braimah}\\
Stony Brook University \\
{\rm ID Number: 115099306}
% \and
% {\rm Author 2}\\
% Stony Brook University
}

% Date
% \date{March 2024}
\date{\today}

% Variables
\def \repoLink{https://github.com/AdebayoBraimah/CSE505}
\def \docLink{https://cse505.readthedocs.io/en/latest/?badge=latest}

% New commands
\newcommand*{\fullref}[1]{\hyperref[{#1}]{\autoref*{#1} \nameref*{#1}}}

\begin{document}
    
    % Title
    \maketitle
        
    \section{Problem \& Plan}
    \label{sec:prob_plan}
    
    \subsection{Problem Description}
    \label{subsec:problem}
    University course planning and understanding of graduation requirements can be a difficult process for new, in-coming students at all levels of one's education. Generally, new students are guided through this process by way of an academic advisor. However, this approach is expensive in both time and personnel (which are usually university faculty) -- especially in the case in which the personnel have to be trained on where to find and understand these graduation requirements. Additionally, in some cases -- the advising can be further complicated by the student's own personal interests (e.g. research focus, specific areas of interests, etc). The proposed solution to this problem would be a knowledge graph(s) augmented generative AI for course recommendation and schedule building. A summary of the inputs and outputs are shown below:

    \subsubsection{Input \& Output}
    \label{subsubsec:in-out}
    
    \textbf{Inputs}:

    \begin{itemize}
        \label{items:inputs}
        \item Major (and minor if specified)
        \item Degree level(s): non-degree seeking, associates, bachelors, masters, doctorate
        \item Current degree progress (e.g. classes already taken)
        \item Preferred classes
        \item Preferred graduation timeline
        \item Knowledge graphs
        \begin{itemize}
            \item Graduation requirements
            \item Department policies (e.g. restrictions on pass/fail courses)
            % Policy links:
            % https://www.cs.stonybrook.edu/sites/default/files/drupalfiles/basicpage/Undergraduate%20Policies.pdf
            % https://www.cs.stonybrook.edu/sites/default/files/drupalfiles/basicpage/Professional%20Ethics.pdf
        \end{itemize}
    \end{itemize}

    \textbf{Outputs}:

    \begin{itemize}
        \label{items:outputs}
        \item Recommended schedules
        \item Course recommendations
        \item Course reviews
    \end{itemize}

    % Requirements
    \subsubsection{Requirements}
    \label{subsubsec:reqs}
    The requirements for this project would include:

    \begin{itemize}
        \item Tools
        \begin{itemize}
            \item Python
            % \item Clingo (First-Order constraint solver)
            \item ErgoAI (Knowledge graph (KG) querying)
            \item LLMs (large language models, multiple are shown -- likely only one will be used)
            \begin{itemize}
                \item LLaMA\cite{touvron2023llama}/LLaMA-2\cite{touvron2023llama2}
                \item Alpaca\cite{alpaca}
                \item Chat-GPT 3.5-turbo\cite{ye2023gpt3.5}
            \end{itemize}
            \item Sub-graph extraction
            \begin{itemize}
                \item Neural State Machine for KBQA\footnote{Knowledge Base Question Answering (KBQA)}\cite{He-WSDM-2021}
            \end{itemize}
        \end{itemize}
        \item Performance evaluation
            \begin{itemize}
                \item Measure the time and accuracy of each query and compare it to Stony Brook University's schedule builder\cite{sched}
            \end{itemize}
        \item Comparison of solutions
            \begin{itemize}
                \item Compare the output of the course schedules and recommendations with Stony Brook University graduation requirements (for select majors).
            \end{itemize}
    \end{itemize}
    
    % example uses
    \subsubsection{Example use-cases}
    \label{subsubsec:example}
    Moreover, the use cases of these solutions from this project are widely applicable to Stony Brook University's undergraduate, and graduate student populations as a whole. For example, these groups of students would find significant utility from this project's solutions:
    \begin{itemize}
        \item An undergraduate computer science student looking to meet the graduation requirements for a combined BS/MS in 5 years.
        \item A MS graduate student in the computer science department looking to satisfy graduation requirements in 2 years, taking 9--12 credits per semester.
        \item A computer science PhD student looking to satisfy course and department policy requirements prior to candidacy.
    \end{itemize}

    Granted the above use-cases were for computer science students -- ideally, most of the Stony Brook University student population would derive some benefit from this project.

    \subsection{State of the art}
    \label{subsec:sota}
    Current state-of-the-art (SOTA) approaches for these problems at Stony Brook University include: the \href{https://you.stonybrook.edu/uaamedia/schedulebuilder/}{schedule builder} and \href{https://classie-evals.stonybrook.edu/}{classie-evals} (for course evaluations). In the case of the schedule builder -- it will mostly help students build a schedule, semester by semester\cite{sched}, but it will not recommend classes nor will it direct students to the course reviews to those classes. Additionally, using the schedule builder requires one to login into \href{https://it.stonybrook.edu/services/solar}{SOLAR}\cite{sched}. Currently, this is the best option for students creating schedules as it is widely available to Stony Brook students.
    In the case of the classie-evals, it does a good job gathering course reviews in one place\cite{class} -- however, that information could be made more available during a student's schedule building process. Furthermore, this is the best option for students deciding on whether to take a course as this information is widely available to all Stony Brook students.
    Current SOTA LLMs \cite{alpaca,touvron2023llama,touvron2023llama2,ye2023gpt3.5} have been shown to hallucinate (i.e. make very untrue statements and assert them as fact). Augmenting these LLMs as shown in \cite{luo2024} using KGs has demonstrated a viable approach to mitigating the hallucination issue that is common with LLMs. The main purpose of using a LLM in this project is to support the feature of gathering and organizing the course recommendation information.

    \subsection{Tasks \& Sub-tasks}
    \label{subsec:tasks}
    Currently, this project has no relation to any external projects (both through adjacent course work and for research purposes). Below are the corresponding tasks and sub-tasks for the project:

    \begin{itemize}
        \item \textbf{Task 1}: Knowledge graph construction (via sub-graph extraction\cite{He-WSDM-2021})
        \begin{itemize}
            \item \textbf{Sub-task 1.1}: Create knowledge graphs of Stony Brook University undergraduate and graduate computer science graduation requirements (including department policies)
            \item \textbf{Sub-task 1.2}: Automate this process (if possible)
        \end{itemize}
        \item \textbf{Task 2}: Build API
        \begin{itemize}
            \item \textbf{Sub-task 2.1}: Use the approach shown in \cite{luo2024} to augment a LLM \cite{alpaca,touvron2023llama,touvron2023llama2,ye2023gpt3.5}
            \item \textbf{Sub-task 2.2}: Task specific fine-tuning of the pre-trained LLM
            \item \textbf{Sub-task 2.3}: Create output knowledge base that can be queried
        \end{itemize}
        \item \textbf{Task 3}: Test \& Evaluate
        \begin{itemize}
            \item \textbf{Sub-task 3.1}: Perform and automate test queries using commonly asked questions
            \item \textbf{Sub-task 3.2}: Evaluate performance (query time and accuracy)
        \end{itemize}
    \end{itemize}

    \subsection{Project plan} 
    \label{subsec:plan}
    
    The repository for the planned code base is located at this public \href{\repoLink}{GitHub repository}. Additionally, the planned timeline of the project is shown below in \fullref{tikz:timeline}, with each set of tasks and sub-tasks (subsection \ref{subsec:tasks}) as checkpoints.
    
    % See this StackOverflow answer for further details:
% https://stackoverflow.com/a/77796855

%Curved Timeline
\begin{tikzpicture}[remember picture, overlay, shift={(8,-3)}]

  %Define control points for the first S-shaped curve
  \coordinate (start) at (-4, 0);
  \coordinate (control1) at (6, -3);
  \coordinate (middle) at (0, -7);
  \coordinate (end1) at (-6, -11);

  % Draw the first S-shaped curve
  \draw(start) .. controls (control1) .. (middle) .. controls (middle) and (end1) .. (end1);

  % Optional: Add labels to the control points for the first curve
  %\filldraw [purple] (start) circle (2pt) node[below] {Start};
  %\filldraw [red] (control1) circle (2pt) node[left] {Control 1};
  %\filldraw [red] (middle) circle (2pt) node[above] {Middle};
  %\filldraw [red] (end1) circle (2pt) node[below] {End 1};

  % Define control points for the second S-shaped curve
  \coordinate (control3) at (8, -2.5);
  \coordinate (middle2) at (4, -6.5);
  \coordinate (end2) at (-3, -14);

  % Draw the second S-shaped curve with an arrow at the end
  \draw (start) .. controls (control3) .. (middle2) .. controls (middle2) and (end2) .. (end2);

  % Optional: Add labels to the control points for the second curve
  %\filldraw [blue] (control3) circle (2pt) node[right] {Control 3};
  %\filldraw [blue] (middle2) circle (2pt) node[below] {Middle 2};
  %\filldraw [blue] (end2) circle (2pt) node[above] {End 2};

  %General coordinates
  \coordinate (arrow1) at ($(end1) + (-1, +1)$);
  \coordinate (arrow2) at ($(end2) + (+1, -1)$);
  \coordinate (arrow3) at ($(arrow2) + (-5.3, 0)$);

  % Optional: Add labels to the control points for the arrow head
  %\filldraw [green] (arrow1) circle (2pt) node[below] {Arrow 1};
  %\filldraw [green] (arrow2) circle (2pt) node[above] {Arrow 2};
  %\filldraw [green] (arrow3) circle (2pt) node[above] {Arrow 3};

  %Draw the arrow head
  \draw (end1) -- (arrow1) -- (arrow3) -- (arrow2) -- (end2);

 %Shade the region between the two curves
 \begin{scope}
    \shade[bottom color=blue!10, top color=blue!60, opacity=0.8] 
      (arrow3) -- (arrow2) -- (end2) -- (middle2) .. controls (control3) .. (start) .. controls (control1) .. (middle) .. controls (middle) and (end1) .. (end1) -- (arrow1) -- (arrow3) -- cycle;
  \end{scope}

  % Define control points for the curved path
  \coordinate (pathControl) at ($(control1)!0.5!(control3)$);
  \coordinate (pathMiddle) at ($(middle)!0.5!(middle2)$);
  \coordinate (pathEnd) at (arrow3);
  \coordinate (pathEnd2) at ($(end1)!0.5!(end2)$);

  % Draw the curved path
  %\filldraw [yellow] (pathControl) circle (2pt) node[below] {Path Control};
  %\filldraw [yellow] (pathMiddle) circle (2pt) node[below] {Path Middle};
  %\filldraw [yellow] (pathEnd) circle (2pt) node[below] {Path End};
  %\filldraw [yellow] (pathEnd2) circle (2pt) node[below] {Path End2};
  %\draw[red] (start) .. controls (pathControl) .. (pathMiddle) .. controls (pathMiddle) and (pathEnd) .. (pathEnd2);
  %\draw[blue] (pathEnd2) -- (pathEnd);

  % Draw ovals at the starting point
  \draw[color = red, fill = red, opacity = 0.8, line width=0.01cm] ($(start) + (0.806, -0.2)$) ellipse [x radius=0.098cm, y radius=0.0225cm];
  \draw[color = orange, fill = orange, opacity = 0.8, line width=0.01cm] ($(start) + (4.2, -1.074)$) ellipse [x radius=0.436cm, y radius=0.155cm];
  \draw[color = cyan, fill = cyan, opacity = 0.8, line width=0.01cm] ($(start) + (6.85, -1.84)$) ellipse [x radius=0.77cm, y radius=0.27cm];
  \draw[color = green, fill = green, opacity = 0.8, line width=0.01cm] ($(start) + (9.23, -3.55)$) ellipse [x radius=1.135cm, y radius=0.5cm];
  \draw[color = teal, fill = teal, opacity = 0.8, line width=0.01cm] ($(start) + (7.31, -5.8)$) ellipse [x radius=1.22cm, y radius=0.65cm];
  \draw[color = blue, fill = blue, opacity = 0.8, line width=0.01cm] ($(start) + (5.02, -7.7)$) ellipse [x radius=1.72cm, y radius=0.75cm];
  \draw[color = purple, fill = purple, opacity = 0.8, line width=0.01cm] ($(start) + (2.1, -10.111)$) ellipse [x radius=2.36cm, y radius=0.95cm];

  % Draw ovals around the existing ovals with the same ratios
  \draw[color=red, line width=0.005cm] ($(start) + (0.806, -0.2)$) ellipse [x radius=0.098cm*1.575, y radius=0.0225cm*1.575];
  \draw[color=orange, line width=0.005cm] ($(start) + (4.2, -1.074)$) ellipse [x radius=0.436cm*1.5, y radius=0.155cm*1.5];
  \draw[color = cyan, line width = 0.005cm] ($(start) + (6.85, -1.84)$) ellipse [x radius=0.77cm*1.3, y radius=0.27cm*1.3];
  \draw[color = green, line width = 0.005cm] ($(start) +(9.23, -3.55)$) ellipse [x radius=1.135cm*1.3, y radius=0.5cm*1.3];
  \draw[color = teal, line width = 0.005cm] ($(start) + (7.31, -5.8)$) ellipse [x radius=1.22cm*1.3, y radius=0.65cm*1.3];
  \draw[color = blue, line width = 0.005cm] ($(start) + (5.02, -7.7)$) ellipse [x radius=1.72cm*1.25, y radius=0.75cm*1.25];
  \draw[color = purple, line width = 0.005cm] ($(start) + (2.1, -10.111)$) ellipse [x radius=2.36cm*1.2, y radius=0.95cm*1.2];

  % Draw perpendicular lines going up 3cm
  \draw[color = red, dashed, opacity = 1] ($(start) + (0.806, -0.2)$) -- ++(90:3.5cm) node[right, scale = 0.9, color = red] {Mar. 15, 2024};
  \draw[color = orange, dashed, opacity = 1] ($(start) + (4.2, -1.074)$) -- ++(90:3.5cm) node[right, scale = 0.9, color = orange] {Mar. 22, 2024};
  \draw[color = cyan, dashed, opacity = 1] ($(start) + (6.85, -1.84)$) -- ++(90:3.5cm) node[right, scale = 0.9, color = cyan] {Mar. 29, 2024};
  \draw[color = green, dashed, opacity = 1] ($(start) + (9.23, -3.55)$) -- ++(-90:3.5cm)  node[right, scale = 0.9, color = green] {April 05, 2024};
  \draw[color = teal, dashed, opacity = 1] ($(start) + (7.31, -5.8)$) -- ++(-90:3.5cm) node[right, scale = 0.9, color = teal] {April 19, 2024};
  \draw[color = blue, dashed, opacity = 1] ($(start) + (5.02, -7.7)$) -- ++(-90:4.5cm) node[right, scale = 0.9, color = blue] {April 26, 2024};
  \draw[color = purple, dashed, opacity = 1] ($(start) + (2.1, -10.111)$) -- ++(-90:4.5cm) node[right, scale = 0.9, color = purple] {May 03, 2024};

  % Add text under the years
  \node[right, color = black, scale = 1] (1) at ($(start) + (0.87, -0.2)  + (90:2.8cm)$) {\textbf{Phase 1}:};
  \node[right, color = black, scale = 1] (2) at ($(start) + (0.87, -0.32)  + (90:2.4cm)$) {1.1 \& 2.1};
  \node[right, color = black, scale = 1] (3) at ($(start) + (4.27, -1.074)  + (90:2.8cm)$) {\textbf{API}: (begin)};
  \node[right, color = black, scale = 1] (4) at ($(start) + (4.27, -1.074) + (90:2.4cm)$) {1.2, 2.1 -- 2.3};
  \node[right, color = black, scale = 1] (5) at ($(start) + (6.95, -1.84) + (90:2.8cm)$) {\textbf{API}: (end)};
  \node[right, color = black, scale = 1] (6) at ($(start) + (6.95, -1.84)+ (90:2.4cm)$) {1.2, 2.1 -- 2.3};
  \node[right, color = black, scale = 1] (7) at ($(start) + (6.95, -1.84) + (90:2cm)$) {(cont.)};
  \node[right, color = black, scale = 1] (8) at ($(start) + (9.38, -3.55) + (-90:2.4cm)$) {\textbf{Prototype}:};
  \node[right, color = black, scale = 1] (9) at ($(start) + (9.38, -3.55) + (-90:2.8cm)$) {2.2, 2.3, 3.1};
  \node[right, color = black, scale = 1] (10) at ($(start) + (7.36, -5.8) + (-90:2cm)$) {\textbf{Phase 2}:};
  \node[right, color = black, scale = 1] (11) at ($(start) + (7.36, -5.8) + (-90:2.4cm)$) {Prototype};
  \node[right, color = black, scale = 1] (12) at ($(start) + (7.36, -5.8) + (-90:2.8cm)$) {2.2, 2.3, 3.1};
  \node[right, color = black, scale = 1] (13) at ($(start) + (5.09, -7.7) + (-90:3cm)$) {\textbf{Test/Eval}:};
  \node[right, color = black, scale = 1] (14) at ($(start) + (5.09, -7.7) + (-90:3.4cm)$) {3.1, 3.2};
  \node[right, color = black, scale = 1] (15) at ($(start) + (5.09, -7.7) + (-90:3.8cm)$) {1.2 (if time)};
  \node[right, color = black, scale = 1] (16) at ($(start) + (2.25, -10.111) + (-90:3.4cm)$) {\textbf{Due Date}};
  \node[right, color = black, scale = 1] (17) at ($(start) + (2.25, -10.111) + (-90:3.8cm)$) {Submit project};

  % Draw a box around the text nodes with the same opacity
  \node[draw, line width = 0.005cm, rounded corners, scale = 0.85, fit={(1) (2)}] {};
  \node[draw, line width = 0.005cm, rounded corners, scale = 0.85, fit={(3) (4)}] {};
  \node[draw, line width = 0.005cm, rounded corners, scale = 0.85, fit={(5) (6) (7)}] {};
  \node[draw, line width = 0.005cm, rounded corners, scale = 0.85, fit={(8) (9)}] {};
  \node[draw, line width = 0.005cm, rounded corners, scale = 0.85, fit={(10) (11) (12)}] {};
  \node[draw, line width = 0.005cm, rounded corners, scale = 0.85, fit={(13) (14) (15)}] {};
  \node[draw, line width = 0.005cm, rounded corners, scale = 0.85, fit={(16) (17)}] {};
\end{tikzpicture}

% \end{document} \label{tikz:timeline} \newpage

    % Design docs
    % See this link for further details:
    % https://github.com/aws/aws-sam-cli/blob/develop/designs/intrinsics_design.md
    \section{Design}
    \label{sec:design}

    \subsection{API}
    \label{subsec:api}

    The driver program of this project can be run as shown from the command line:

    {\tt{./app.py}}

    Note, that file permissions may need to be changed for the file to run.

    The outputs of the file are the results of the test query.

    The contents of the {\tt{app.py}} file are as follows: \\

    \begin{python}
    import os
    from src.ergoai.ergoai import query_ergoai
    
    test_file: str = os.path.abspath("foo.ergo")
    query: str = "?Subject[?Property->?Object]"
    
    query_ergoai(knowledge=test_file, query=query)
    \end{python}

    \subsubsection{Implementation Details}
    \label{subsubsec:implementation}

    The implementation details of this project (at this point in time) mostly include simple test queries to ErgoAI.
    

    \subsection{Third Party Libraries}
    \label{subsec:thirdparty}

    At this point in time -- no third party libraries have been implemented in the design. However, third party libraries will be included later in the development process (e.g. knowledge graph creation, and the course recommendations).

    \subsection{Documentation}
    \label{subsec:docs}

    The documentation of this project is contained in the {\tt{doc}} folder. Documentation was also written in reStructured Text ({\tt{.rst}}) files, and built using python via \href{https://www.sphinx-doc.org/en/master/}{Sphinx} ({\tt{HTML}} documentation can be found in {\tt{doc/build/html/index.html}}). The {\tt{HTML}} documentation can found online \href{\docLink}{here} (recommended method of viewing this documentation).
    
    % \section{Implementation}
    % \label{sec:implement}
    
    % \section{Testing \& Evaluation}
    % \label{sec:test_eval}
    
    \section{Acknowledgements}
    \label{sec:ack}

    % TODO:
    %   - Update acknowledgements section
    %   - Include section that tracks changes or
    %       at least mentions what was changed 
    %       and why.

    % Add newpage here to separate references from the
    % main text body
    \newpage
    
    % Bibliography & References
    \bibliographystyle{acm}
    \bibliography{refs}
    
    % \printbibliography

    \newpage
    
    \section{Appendix}
    \label{sec:appendix}

    \subsection{Changelog (Change Log)}
    \label{subsec:change}

    % LLM
    % Project scope
    % Knowledge graph extraction
    % Course review/recommendation
    % Project external dependencies (link to documentation)
    % Python dependencies (link to relevant docs/websites)
    % Mention web-scraper

\end{document}
